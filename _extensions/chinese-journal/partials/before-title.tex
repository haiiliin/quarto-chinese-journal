\usepackage{xeCJK}
\usepackage{fancyhdr}
\usepackage{float}
\usepackage{geometry}
\usepackage{indentfirst}
\usepackage{multicol}
\usepackage[marginal]{footmisc} % 脚注首行不缩进。
\usepackage{pbalance}

\geometry{left=1.6cm,right=1.8cm,top=2.0cm,bottom=1.7cm,headheight=0.6cm}
\setlength{\parindent}{2em}
\linespread{1.3}

% 设置字体大小
\newcommand{\chuhao}{\fontsize{42pt}{\baselineskip}\selectfont}
\newcommand{\xiaochuhao}{\fontsize{36pt}{\baselineskip}\selectfont}
\newcommand{\yihao}{\fontsize{28pt}{\baselineskip}\selectfont}
\newcommand{\erhao}{\fontsize{21pt}{\baselineskip}\selectfont}
\newcommand{\xiaoerhao}{\fontsize{18pt}{\baselineskip}\selectfont}
\newcommand{\sanhao}{\fontsize{15.75pt}{\baselineskip}\selectfont}
\newcommand{\sihao}{\fontsize{14pt}{\baselineskip}\selectfont}
\newcommand{\xiaosihao}{\fontsize{12pt}{\baselineskip}\selectfont}
\newcommand{\wuhao}{\fontsize{10.5pt}{\baselineskip}\selectfont}
\newcommand{\xiaowuhao}{\fontsize{9pt}{\baselineskip}\selectfont}
\newcommand{\liuhao}{\fontsize{7.875pt}{\baselineskip}\selectfont}
\newcommand{\qihao}{\fontsize{5.25pt}{\baselineskip}\selectfont} 

% 页眉页脚设置
\fancypagestyle{plain}{
    \fancyhf{}
    \setboolean{first}{false}
    \lhead{第~$volume$~卷\quad 第~$number$~期\\
    \scriptsize{$cyearmonth$}}
    \chead{\centering{$cjournal$\\
    \scriptsize{\textbf{$ejournal$}}}}
    \rhead{Vol. $volume$, No. $number$\\
    \scriptsize{$eyearmonth$}}
    \lfoot{}
    \cfoot{}
    \rfoot{}
}

% 首页后根据奇偶页不同设置页眉页脚
% R,C,L分别代表左中右,O,E代表奇偶页
\newboolean{first}%定义一个布尔变量用于判断是否为首页
\setboolean{first}{true}%设定fist变量初值为true
\pagestyle{fancy}{
    \fancyhf{}
    \fancyhead[RE]{第~$volume$~卷}
    \fancyhead[CE]{$cjournal$}
    \fancyhead[LE,RO]{\thepage}
    \fancyhead[CO]{$for(cauthor/allbutlast)$$cauthor$,$endfor$$for(cauthor/last)$$cauthor$$endfor$}
    \fancyhead[LO]{第~$number$~期}
    \lfoot{}
    \cfoot{}
    \rfoot{}
}
\thispagestyle{plain}

% 脚注不编号
\renewcommand{\thefootnote}{}

% 标题命令
\renewcommand\abstractname{摘要}  
\renewcommand\refname{参考文献}  
\renewcommand\figurename{图}  
\renewcommand\tablename{表}

% 表格设置
\usepackage{float,stfloats}
$if(twocolumn)$
    \def\env{$if(table-env)$$table-env$$else$figure$endif$}
    \makeatletter
    \let\oldlt\longtable
    \let\endoldlt\endlongtable
    \def\longtable{\@ifnextchar[\longtable@i \longtable@ii}
    \def\longtable@i[#1]{\begin{\env}[$if(table-pos)$$table-pos$$else$!htb$endif$]
    \onecolumn
    \begin{minipage}{0.5\textwidth}
    \oldlt[#1]
    }
    \def\longtable@ii{\begin{\env}[$if(table-pos)$$table-pos$$else$!htb$endif$]
    \onecolumn
    \begin{minipage}{0.5\textwidth}
    \oldlt
    }
    \def\endlongtable{\bottomrule\endoldlt
    \end{minipage}
    \twocolumn
    \end{\env}}
    \makeatother
$endif$
